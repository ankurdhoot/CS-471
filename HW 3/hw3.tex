\documentclass[11pt,a4paper]{article}
\usepackage{graphicx}
\usepackage{titling}
\usepackage{pdfpages}
\usepackage{amsmath}
\usepackage{amssymb}
\usepackage{MnSymbol}
\begin{document}
\author{Ankur Dhoot}
\title{CS 471 HW 3}
\maketitle

\section*{Q1}

\paragraph*{(b)}
P(x $|$ y) = P(x,y)/P(y) = $\sum\limits_{Z} P(x,y,z)/P(y)$ = $\sum\limits_{Z} P(x,z | y)$  

where the second equality follows because the joint distribution must be consistent with the marginal.

\section*{Q2}
\paragraph*{a}
Symmetry: $(X \upmodels Y \mid Z) \Rightarrow (Y \upmodels X \mid Z)$

We have $P(X|Y,Z) = P(X|Z)$.

\begin{equation*}
	\begin{aligned}
	P(Y|X,Z) = \dfrac{P(Y,X,Z)}{P(X,Z)} = \dfrac{P(X|Y,Z)P(Y,Z)}{P(X,Z)} \\= \dfrac{P(X|Z)P(Y,Z)}{P(X|Z)P(Z)} = \dfrac{P(Y,Z)}{P(Z)} = P(Y|Z) \\ 
	\therefore (Y \upmodels X \mid Z)
	\end{aligned}
\end{equation*}

Decomposition: $(X \upmodels Y,W \mid Z) \Rightarrow (X \upmodels Y \mid Z)$

We have $P(X,Y,W|Z) = P(X|Z)P(Y,W|Z)$
\begin{equation*}
	\begin{aligned}
	P(X,Y|Z) = \sum\limits_{w} P(X,Y,w | Z) = \sum\limits_{w} P(X|Z)P(Y,w|Z) \\= P(X|Z)\sum\limits_{w}P(Y,w|Z) = P(X|Z)P(Y|Z) \\
	\therefore (X \upmodels Y \mid Z)
	\end{aligned}
\end{equation*}

Weak Union: $(X \upmodels Y,W \mid Z) \Rightarrow (X \upmodels Y \mid Z,W)$

We have $P(X|Y,W,Z) = P(X|Z)$ and by Decomposition $P(X|W,Z) = P(X|Z)$
\begin{equation*}
	\begin{aligned}
	P(X,Y|W,Z) = P(X|Y,W,Z)P(Y|W,Z) = \\ P(X|Z)P(Y|W,Z) = P(X|W,Z)P(Y|W,Z) \\
	\therefore (X \upmodels Y \mid Z,W)
	\end{aligned}
\end{equation*}

Contraction: $(X \upmodels Y \mid Z,W)$ \& $(X \upmodels Y \mid Z) \Rightarrow (X \upmodels Y,W \mid Z)$

We have $P(X|Y,W,Z) = P(X|Y,Z)$ and $P(X|Y,Z) = P(X|Z)$
\begin{equation*}
	\begin{aligned}
	P(X,Y,W|Z) = P(X|Y,W,Z)P(Y,W|Z) = \\ P(X|Y,Z)P(Y,W|Z) = P(X|Z)P(Y,W|Z) \\
	\therefore (X \upmodels Y,W \mid Z)
	\end{aligned}
\end{equation*}

\paragraph*{(b)}

\section*{Q3}

\paragraph*{(a)}
$P(y|x_{1},x_{2}) = \dfrac{P(x_{1},x_{2},y)}{P(x_{1},x_{2})} = \dfrac{P(x_{1},x_{2}|y)P(y)}{P(x_{1},x_{2})}$

Only the second set of distributions can be used:


It's clear we can calculate the desired quantity using the second set of distributions ($P(x_{1},x_{2}), P(y), P(x_{1},x_{2}|y)$) since these are exactly the terms that appear above. 

The first and third set of distributions require the additional assumption that $(X_{1} \upmodels X_{2} \mid Y)$ since $P(x_{1},x_{2}|y) \neq P(x_{1}|y)P(x_{2}|y)$ in general. We'll show in (b) how to use this assumption to calculate the desired quantity.

The fourth set of distributions provides us no way of knowing $P(y)$ nor $P(x_{1},x_{2})$ since $P(x_{1},x_{2}) \neq P(x_{1})P(x_{2})$ in general.

The fifth set of distributions also provides us no way of knowing $P(y)$ nor $P(x_{1},x_{2})$ since $P(x_{1},x_{2}) \neq P(x_{1})P(x_{2})$ in general nor $P(x_{1},x_{2}|y)$ since $P(x_{1},x_{2}|y) \neq P(x_{1}|y)P(x_{2}|y)$ in general.

\paragraph*{(b)}
If we are told that $(X_{1} \upmodels X_{2} \mid Y)$, then $P(X_{1},X_{2}|Y) = P(X_{1}|Y)P(X_{2}|Y)$. \newline

Thus $P(y|x_{1},x_{2}) = \dfrac{P(x_{1},x_{2},y)}{P(x_{1},x_{2})} = \dfrac{P(x_{1},x_{2}|y)P(y)}{P(x_{1},x_{2})} = \dfrac{P(x_{1}|y)P(x_{2}|y)P(y)}{P(x_{1},x_{2})}$

Obviously, the second distribution can still be used to calculate the desired quantity since that distribution requires no additional assumptions.

The first distribution ($P(x_{1}, x_{2}, P(y), P(x_{1}|y), P(x_{2}|y)$) can be used since it contains all the components in the final equality above. 

The third distribution ($P(x_{1}|y), P(x_{2}|y), P(y)$) can also be used with a little bit more work:

We have the components for the numerator in the final equality above so we need to still compute the denominator.

$P(x_{1},x_{2}) = \sum\limits_{y} P(x_{1}, x_{2}, y) = \sum\limits_{y} P(x_{1}, x_{2} | y)P(y) = \sum\limits_{y} P(x_{1}|y)P(x_{2}|y)P(y)$.

All the terms in the final equality occur in the third distribution. Thus, we can use the third distribution to compute the desired quantity.

The fourth and fifth distributions still can't be used because they provide no way of calculating $P(y)$.

\section*{Q4}
\paragraph*{(a)}

\paragraph*{(b)}

\section*{Q5}

\section*{Q6}
$Q = \sum\limits_{Z_{1}}P(y|x,z_{1})P(z_{1})$
\paragraph*{(a)}
$P(y|x) = \sum\limits_{Z_{1}} P(y,z_{1}|x) = \sum\limits_{Z_{1}} P(y|z_{1},x)P(z_{1}|x)$

Looking at corresponding terms, we see that Q = $P(y|x)$ if $P(z_{1}|x) = P(P(z_{1})$ (i.e $(Z_{1} \upmodels X)$). But the graph implies no such independence, so it's not necessarily the case that the distribution P satisfies ($Z_{1} \upmodels X)$. Thus, we cannot conclude that Q = $P(y|x)$.

\paragraph*{(b)}
Yes. From the graph, we see the following independencies: 

($Y \upmodels Z_{1} \mid X, Z_{2})$) and ($Z_{2} \upmodels X \mid Z_{1}$)

which means $P(Y|Z_{1},Z_{2},X) = P(Y|X,Z_{2})$ and $P(Z_{2}|X,Z_{1}) = P(Z_{2}|Z_{1})$

\begin{equation*}
	\begin{aligned}
	Q = \sum\limits_{Z_{1}}P(y|x,z_{1})P(z_{1}) \\= 
\sum\limits_{Z_{1}} \sum\limits_{Z_{2}} P(y,z_{2}|x,z_{1})P(z_{1}) \\= 
\sum\limits_{Z_{1}} \sum\limits_{Z_{2}} P(y|z_{1},z_{2},x)P(z_{2}|x,z_{1})P(z_{1}) \\= 
\sum\limits_{Z_{1}} \sum\limits_{Z_{2}} P(y|z_{2},x)P(z_{2}|x,z_{1})P(z_{1}) \\= 
\sum\limits_{Z_{1}} \sum\limits_{Z_{2}} P(y|z_{2},x)P(z_{2}|z_{1})P(z_{1}) \\= 
\sum\limits_{Z_{1}} \sum\limits_{Z_{2}} P(y|z_{2},x)P(z_{1},z_{2}) \\= 
\sum\limits_{Z_{2}} P(y|z_{2},x) \sum\limits_{Z_{1}} P(z_{1},z_{2}) \\=
\sum\limits_{Z_{2}} P(y|z_{2},x)P(z_{2}) \\= 
\sum\limits_{Z_{2}} \dfrac{P(x,y,z_{2})P(z_{2})}{P(x,z_{2})}
	\end{aligned}
\end{equation*}

$P(z_{2}) = \sum\limits_{X} \sum\limits_{Y} P(x,y,z_{2})$

$P(x,z_{2}) = \sum\limits_{Y} P(x,y,z_{2})$

Thus, we can compute all terms in the final equality in terms of the given distribution, $P(X,Y,Z_{2})$.

\section*{Q7}








\end{document}